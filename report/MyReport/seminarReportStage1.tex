\documentclass[a4paper,10pt]{article}
\usepackage[utf8]{inputenc}

%opening
\title{\textbf{Virtualization for embedded systems}\\M.Tech. Seminar Report}
\author{\textbf{Sahil Singh}\\143050001\\\\\\Seminar Guide\\\textbf{Prof. Purushottam Kulkarni}\\\\\\\\\\\\Department of Computer Science and engineering\\Indian Institute of Technology Bombay\\Mumbai, Maharashtra}

\date{}

\begin{document}

\maketitle
\newpage
\begin{abstract}
 
\end{abstract}
\tableofcontents

\section{Introduction}
\subsection{Introduction to virtualization}
%\subsection{scope of virtualization}
\subsection{Aim of this seminar}

\section{Motivations for virtualization on x86 platform}
%server consolidation
This section discusses the reasons for virtualizing x86 Hardware, like server consolidation, efficient use of resources, energy constraints.
The contents for this section will be mainly from cs695 course I took in last semester.

\section{Motivations for virtualization on embedded platforms}
This section discusses the reasons for virtualizing a non-x86 hardware, and contrasts it with the motivations for those of virtualization on x86 platform, as discussed in previous section.
Main reasons are the need for providing isolated environments, as in a BYOD scenario, ease of running different kinds of software, i.e. software developed for different OSes.
ARM processors are slowly increasing their spread,and there are already efforts to run them in servers. With such a trend 
%isolated environments, ease of running software of different types, providing different features
\section{Architectural differences from x86}

\section{Unique challenges on non-x86 platforms}

%device table will come here

\section{Classification of virtualization solutions for different devices on a mobile phone}
this section will have a table of devices in a mobile phone, like CPU, camera, etc. and the way to virtualize them.

\section{Virtualization solutions}
this section will discuss the Hypervisors that I have studied during the seminar, under two broad headings of paravirtualization solutions and full virtualization solutions
with a table listing and comparing features of each.
\section{Conclusion}
\end{document}
